% -*- TeX-PDF-mode: t -*-

% Template for ICASSP-2010 paper; to be used with:
%          spconf.sty  - ICASSP/ICIP LaTeX style file, and
%          IEEEbib.bst - IEEE bibliography style file.
% --------------------------------------------------------------------------
\documentclass{article}
\usepackage{spconf,amsmath,graphicx}
\usepackage{url}
\usepackage{verbatim}
\usepackage{subfig}
%\usepackage{microtype}

% Example definitions.
% --------------------
\def\x{{\mathbf x}}
\def\L{{\cal L}}

\usepackage{color}
\newcommand{\FIXME}[2][FIXME]{\textcolor{blue}{\textbf{#1}: \emph{#2}}}

% Title.
% ------
%\title{IMPUTATION OF BEAT-ALIGNED FEATURES AND MUSIC PATTERN LEARNING}
\title{EVALUATING MUSIC SEQUENCE MODELS THROUGH MISSING DATA}
%
% Single address.
% ---------------
%\name{Thierry Bertin-Mahieux\thanks{Thanks to NSERC and some other stuff.}}
%\address{EE dept., Columbia University}
%
% For example:
% ------------
%\address{School\\
%	Department\\
%	Address}
%
% Two addresses (uncomment and modify for two-address case).
% ----------------------------------------------------------
\twoauthors {Thierry Bertin-Mahieux\sthanks{supported in part
    by a NSERC PG scholarship.}, Graham Grindlay} {Columbia
  University\\ LabROSA\\ New York, USA} 
   {Ron J. Weiss\sthanks{supported by NSF grant IIS-0844654 and Institute of
    Museum and Library Services grant LG-06-08-0073-08.}  and Daniel
  P.W. Ellis} 
{New York University / Columbia University\\ MARL / LabROSA\\ New
  York, USA}

\begin{document}
\ninept
%
\maketitle
%
\begin{abstract}
Imputation is cool. It is a well-defined problem
\FIXME[ronw]{but isn't the whole point of the paper
that it isn't to straightforward to evaluate?},
as opposed to segmentation for which ground truth labels are often
very ambiguous.
%
Gives a reasonable benchmark to compare algorithms that claim learning meaningful
temporal patterns on music. Hard to beat benchmarks comparison, for instance linear
prediction. We compare some methods and discuss contradicting error measures. 
We'll track you down if you don't accept this paper.
\end{abstract}
%
\begin{keywords}
Missing data, chroma features, beat imputation, codebook learning
\end{keywords}
%

% NSF thanks
\makeatletter{\renewcommand*{\@makefnmark}{}
\footnotetext{This work is supported by NSF grant
    IIS-0713334 and by a gift from Google, Inc.
    Any opinions, findings and conclusions or
    recommendations expressed in this material are those of the
    authors and do not necessarily reflect the views of the sponsors.}\makeatother}

\section{Introduction}
\label{sec:intro}
As with many classes of time-series data, musical signals contain
substantial amounts of complex structural information.  Given the
complexity of this data, learning structural models of music in an
unsupervised fashion is a particularly challenging task.  For many
applications we are interested in models that capture local patterns
(i.e. ``patches'') in the data.  Finding robust patterns may prove
useful for tasks such as song similarity (recognizing songs with
similar patterns), song segmentation (labeling chorus/verse
structure), and cover song recognition (identifying songs with similar
high-level patterns).  All of these tasks would benefit from patch
models that exhibit high-level musical characteristics while remaining
faithful to the observed signal data.
\FIXME[gg]{Do we need to back this up more?}
%Therefore, we are interested in patterns that not only explain the
%low-level signal, but also contain musical characteristics.  
However, it is unclear how to evaluate the quality of this type of
model.  In this paper, we propose a task based on missing data
imputation and discuss various performance metrics which are sensitive
to musically meaningful aspects of the data.

Imputation refers to a family of techniques used to fill in missing
data entries.  Previous work on audio-related applications has
included speech denoising~\cite{Raj1998}, source
separation~\cite{Reyes-Gomez2005}, and model
evaluation~\cite{Smaragdis2009,Hoffman2010}.  However, to the best of
our knowledge, previous research has only considered at most a single
missing time frame.
%a few missing time-frequency \emph{bins}
%\FIXME[ronw]{this isn't totally fair, some of the missing
%  data speech stuff is based on imputation using HMM/CRF models. E.g. see
%  figure 3 of
%  {\footnotesize \url{http://www.ee.columbia.edu/~mjr59/AISTATS-defspec.pdf}}
%}.
In contrast, we define an imputation task in which many consecutive
time frames are missing.  We refer to this task as \emph{multi-frame}
imputation.  This makes the problem more challenging, but also allows
us to more carefully evaluate the temporal aspects of our
model.  % needs to be reworded?
% mention real-world packet-loss case?

\begin{comment}
and fully masking consecutive beats of music have not been
considered. A real-life analogous problem would be a music stream
signal that is lost for a few seconds. The goal here would be to infer
a reasonable stream to replace the original one. Thus we do not only
want a signal close to the original in a reconstruction sense, but
also a signal that has similar ``musical properties'' to the user
ear. These properties are difficult to define, but we refer to higher
order dependencies present in the signal. For instance, the rate of
note onsets, the number of notes activated at the same time, and other
texture-like elements. We will argue that entropy is a good
approximation of these characteristics.
\end{comment}

Figure \ref{fig:types} shows three matrices with different amounts of
missing data.  As mentioned previously, other researchers
\cite{Smaragdis2009} have worked with the partially missing data for
applications such as bandwidth expansion and source separation (see
Figure~\ref{fig:types} \FIXME[ronw]{I don't think we need this figure
  -- especially since we only focus on the last case, an example of
  which is clearly
  visible in figure 2}).  A slightly more complex problem can be
defined by masking out an entire time frame.  This problem is
illustrated in Figure~\ref{fig:types}.  It is worth noting that due to
our data representation (see Section~\ref{ssec:feats}), the
single-frame imputation problem is somewhat easy.  In contrast, the
problem of multiple missing frames (see Figure~\ref{fig:types}) can be
made arbitrarily difficult.

\begin{comment}
a few frequency bins are hidden and can be
recovered from their neighborhood, both in time and frequency. Masking
one full beat is a natural extension, and the solution is easy as
music notes are often sustained over many beats. Two or three missing
beats remain solvable as music bare many repetitions. One can usually
find a similar section from an unmasked part of the song (think about
a repeated chorus for instance).  The real challenge arises from
masking many consecutive beats.
\end{comment}


Music is highly structured in time and as such is particularly suitable
for the multi-frame imputation task.
In addition to musicological implications,
large-scale models of music data have numerous commercial
applications, including recommendation systems, digital rights
management, and creative tools.  

\FIXME[ronw]{Do we need to describe the features before presenting
  this figure?}  
Figure \ref{fig:basic} shows an example of
multi-frame imputation of music data.  Linear regression, which
explicitly minimizes Euclidean distance, is used to impute the missing
data in Figure~\ref{fig:basic}.  However, it is clear from visual
inspection that linear regression yields an overly-smooth
reconstruction and is unable to predict the temporal evolution within
the missing section.
%
A more sophisticated signal model which better leverages the
surrounding context and makes use of long-term temporal information
such as knowledge of repetitions within a song is necessary to achieve
a musically coherent reconstruction.
%
Unfortunately, simple algorithms such as linear regression
are often prone to identifying poor solutions, despite optimizing
metrics such as Euclidean distance.
%
In this paper, we argue that a more extensive set of metrics is needed
to properly evaluate a model's ability to predict musical sequences.
%more for proper evaluation.


\section{TASK DEFINITION}
\label{sec:task}

\subsection{DATA AND FEATURES}
\label{ssec:feats}
We represent music using chroma features \cite{Ellis2007a}, which
record the intensity associated with each of the 12 semitones
(e.g. piano keys) across all octaves.  One can see them as a very
coarse %and noisy
and low dimensional music transcription which captures the
harmonic content of the signal while ignoring information related to
timbre and instrumentation.
We use an online API\footnote{\url{http://developer.echonest.com}} %
%\cite{EchoNest} 
that computes
chroma from a constant-Q spectrogram.
%
Music naturally evolves over a time scale expressed in beats, as
opposed to the fixed length frames commonly used in other audio
processing applications.
We therefore follow
\cite{Ellis2007a} and
resample the features so that each chroma vector spans one beat.
% ronw: this sentence is unnecessary
% Note
%that music beat tracking algorithms perform relatively well
%\cite{Davies2007} even if the task is not considered solved.

Beat-aligned chroma features have been used for cover song recognition
\cite{Ellis2007a} and segmentation \cite{Weiss2010}. In our previous work
\cite{Bertin-Mahieux2010a}, we attempt to learn music patterns in that
representation.
% Results on multi-frame imputation are done on
In the following sections we report experimental results over 
a random
subset of $5000$ songs from the morecowbell.dj dataset
\cite{Bertin-Mahieux2010a}.


\begin{figure}[t]
\begin{center}
\includegraphics[width=.7\columnwidth]{type_imputation}
\end{center}
\caption{Imputation types.
\label{fig:types}}
\end{figure}

\begin{figure}[t]
\begin{center}
\includegraphics[width=.95\columnwidth]{basic}
\end{center}
\caption{$15$ beats imputation example, rows are 1) original 2) original masked
3) reconstruction using a linear transform of one previous beat.
\label{fig:basic}}
\end{figure}

% IPYTHON COMMAND TO RECREATE ABOVE FIGURE
%btchroma2 = sio.loadmat('/home/thierry/Columbia/covers80/coversongs/covers32kENmats/john_lennon+Double_Fantasy+05-I_m_Losing_You.mp3.mat')['btchroma']
%p1=185;p2=p1+15;mask=np.ones(btchroma2.shape);mask[:,p1:p2]=0.
%evaluation.plot_oneexample(btchroma2,mask,p1,p2,methods=['lintrans'],methods_args=[{'win':1}],measures=('eucl','kl','dent'),plotrange=(p1-10,p2+70))


%\subsection{MEASURES}
\subsection{EVALUATION METRICS}
\label{ssec:measures}
Euclidean distance is a natural choice for reconstruction and encoding
tasks. However, we saw from Figure \ref{fig:basic} that it can favor
unsatisfying results for multi-frame imputation. We devote a great
part of this work to exploring metrics that measure other aspects of
the reconstruction: for instance, a metric that would reward
preserving the level of granularity of the original data in Figure
\ref{fig:basic}.

We first consider Mahalanobis distances of the form $d_p =
|x_1-x_2|^p$,
\FIXME[ronw]{I don't think Mahalanobis distance is the right term here
  -- there is no covariance weighting here.  ``Euclidean'' or
  ``generalized Euclidean'' might be
  safer even though ``Euclidean'' implies that $p=2$.}
 especially with $p \leq 2$. Figure \ref{fig:measures}
illustrates the effect of these measures on one dimensional data.  The
greyed rectangle represents the case where a reconstruction is
considered valid if it is between some $\delta$ of the original and
wrong otherwise.  Mahalanobis distance approximate this case with an
increasingly smaller $\delta$ as $p \rightarrow 0$.
\FIXME[ronw]{elaborate on this, it's kind of unclear}

Let's consider these cases on the toy example of Figure
\ref{fig:square}.  The square wave of period $1$ could be one row of
our missing data for instance.  We approximate it by two signals: the
same square wave translated by a quarter of its period and the average
function (constant at $0.5$). The average reconstruction errors on
$[0,1]$ for the translated wave, with any $d_p$, is $0.5$.  For the
average and $d_2$, this error is $0.25$.  Hence, the average function
seems the most accurate. However, with $d_1$, the errors are both
equal to $0.5$. With $d_{1/2}$, errors are now $0.5$ and
$0.71$. Hence, if the translated square wave should be a better
approximation, Euclidean distance is misleading.
\FIXME[ronw]{This paragraph is confusing and should be rewritten.}
%Vasicek estimator \cite{Learned-Miller2003}.  

The phenomena above explain why reconstruction such as in Figure
\ref{fig:basic} ($3$rd row) or in Figure \ref{fig:avgnnrand} ($3$rd
row) often result from powerful algorithms. Learning an overly
smoothed solution and avoiding extreme values is rewarded by some
metrics. $d_{1/2}$ seems to be a solution from the toy example in
Figure \ref{fig:square}, but we will see in Section \ref{sec:exp} that
it is not really the case in practice.  Similarly, Kullback-Leibler
(KL), conditional entropy and Jensen difference \cite{Michel1994} are
two measures based on entropy, thus should not behave necessarily in
the same way as $d_p$.  Once again, it is not the case in practice. We
now look into other measures that would differ from $d_2$.  Upon
visual inspection, they should reward granularity which is missing in
Figure \ref{fig:basic} for instance. Delta chromas are the finite
difference along the time axis of the chromas. Deltas of over-smoothed
solution are closed to $0$ whereas features of typical songs have more
variation over time.
\textbf{ddiff} measures the absolute difference between the
sum of absolute values of the deltas for the original data and its
reconstruction.

Another idea is to look at the histogram of values and compute its
entropy. Once again, sustained solutions should measure differently
than the original.  Absolute normalize difference entropy
(\textbf{D-ENT}) \cite{Mentzelopoulos2004} between two vectors is
computed as follow: we discretize $(0,1)$ in $10$ bins and create the
normalized histogram of values.  The entropy of each of the bins
$b_{bi}$ is $e_b = - log_2 b_b$ for each of the two vectors
$i$. Finally:
\[
\mbox{D-ENT} = \left( \Sigma_b \frac{e_{b1} - e_{b2}}{e_{b1}} \right) / (\mbox{\# bins})
\]
Note that D-ENT is not symmetric. In our experiment, the first vector
is the original one.  If we look at Figure \ref{fig:avgnnrand}, the
reconstruction with the lowest D-ENT is using nearest neighbor and not
averaging as with Euclidean distance.

Note that we tried other entropy related measures,
i.e. Kullback-Leibler divergence (KL) and Jensen difference
\cite{Michel1994}. As for $d_{1/2}$, they behave in a surprisingly
similar fashion as the Euclidean distance.  We explained above why
Euclidean distance justifies disappointing reconstructions. At the
same time, we do not argue that we should ignore or replace
it. Euclidean distance measures reconstruction in a fundamental
way. We believe we need a set of measures to quantify the quality of
music patterns, Euclidean distance being one of them. In the next
section, we investigate which measures behave in a similar way, thus
helping us to decide which ones are useful to report.

\begin{figure}[t]
\begin{center}
\includegraphics[width=.95\columnwidth]{avg_nn_rand}
\end{center}
\caption{Same beat imputation example as Figure \ref{fig:basic}, 
rows are 1) original 2) original masked
3) reconstruction using the average of nearby beats 4) using
nearest neighbor 5) using random.
\label{fig:avgnnrand}}
\end{figure}


\begin{figure}%
\centering
\subfloat[]{%
\label{fig:measures}%
\includegraphics[width=0.475\columnwidth]{measures}} 
\hspace{0.1cm}
\subfloat[]{%
\label{fig:square}%
\includegraphics[width=0.475\columnwidth]{square}}%
\hspace{8pt}%
\caption{(a) Effect of different measures on one-dimensional data. (b)
  Reconstruction error between a square wave and two approximations, a
  square wave translated by a quarter of the period, and the average
  function. Average error between original and translated wave is
  always $0.5$ for any Mahalanobis measure $d_p$ on $[0,1]$.  For the
  average function, the errors are $0.25$, $0.5$ and $0.71$ for $d_2$,
  $d_1$ and $d_{1/2}$ respectively.}%
\label{fig:two_measures}
\end{figure}


\begin{comment}
  Reconstruction error between a square wave and two approximations, a
  square wave translated by a quarter of the period, and the average
  function. Average error between original and translated wave is
  always $0.5$ for any Mahalanobis measure $d_p$ on $[0,1]$.  For the
  average function, the errors are $0.25$, $0.5$ and $0.71$ for $d_2$,
  $d_1$ and $d_{1/2}$ respectively.
\end{comment}

\section{EXPERIMENTS}
\label{sec:exp}
We present results from different algorithms for imputing masked
beats.  Features are beat-aligned chromas and the dataset is made of
$5000$ songs from the Cowbell dataset
\cite{Bertin-Mahieux2010a}. Songs has to be of length at least $100$
beats.  The masked part is chosen at random, excluding the first and
last $30$ beats.

\subsection{ALGORITHMS}
\label{ssec:algo}
As multi-frame imputation has not been studied before, we report
results on a variety of benchmark algorithms. We realize that such
methods have a low probability of success at solving the task, but
their relative performance is of interest.  Simple methods include
\textbf{random} where each chroma bin is drawn from a uniform $[0,1)$
distribution (Figure \ref{fig:avgnnrand}, $5$th row).  One can also
pick a beat at random from the song, or take the average of all
visible beats. Taking the \textbf{average} of the beats within a certain
range of the missing patch (Figure \ref{fig:avgnnrand}, $3$rd row)
create a smooth reconstruction, but still solves the case of
sustained notes.

Our first trained model is linear interpolation (called \textbf{linear
  transform} \FIXME[ronw]{Um, these two aren't the same...  it sounds
  like you're actually training a linear transform  from the previous
  frames onto the missing one, not just finding the point that best
  fits the line that connects the visible beats on either side of the
  mask (i.e. linear interpolation).  Or am I missing something?}
in the text) where we use $N$ previous beats to predict
the next one.  As it explicitly learns to minimize Euclidean distance
on the visible data, this methods performs very well under certain
metrics (See Figure \ref{fig:basic}). From our experiments, $N=1$ or
$N=2$ work best.

Nearest neighbor ($\mathbf{1}$\textbf{NN}) is an appropriate tehcnique
to take advantage of the repetitions within a song. By looking at the
nearby beats of the missing data, we can impute a reconstruction by
spotting a similar neighborhood. Note that instead of using the
visible part of the song, one can use a \textbf{codebook} made of
other songs.  $k$NN with $k>1$ is ongoing research.

Previous methods rely on local information which should not be enough
to solve multi-frame imputation. HMM or NMF should better leverage
long term dependencies and nearby information.  We experiment with
shift invariant probabilistic latent component analysis
(\textbf{SIPLCA}) \cite{Smaragdis2009,Weiss2010}, a probabilistic
version of NMF. SIPLCA learn how to encode the signal as overlapping
patches. The missing data is inferred by maximizing expectation.
Problems arise when the number of missing beats is larger than the patch
size. SIPLCA can activate patches or not without any penalty. Thus, we
regularize the the activation by applying a second level of SIPLCA.
Still, results are not impressive and improvements are part of our
ongoing research. We refer the interested reader to our code for more
details.



\subsection{RESULTS}
\label{ssec:results}
As we mentioned in Subsection \ref{ssec:measures}, many metrics reward
reconstructions in a very similar fashion. The Euclidean distance is
often the default choice, and we will see that many others tend to agree
with it. At the same time, measures like delta difference and D-ENT
offer another perspective. We try to rationalize this idea by
measuring the correlation between our metrics. Pearson's
correlation is defined as
\[  \rho_{X,Y}=\mathrm{corr}(X,Y)={\mathrm{cov}(X,Y) \over \sigma_X
    \sigma_Y} ={E[(X-\mu_X)(Y-\mu_Y)] \over \sigma_X\sigma_Y}
\]
$\rho_{X,Y}$. Remember that $-1 \leq \rho_{X,Y} \leq 1$, an absolute
value close to $1$ meaning high correlation. $\rho_{X,Y}$ is computed
for many error measures based on results obtained by $3$ methods on
two multi-frame imputation tasks. See results in Table
\ref{tab:corrs}. Delta difference and D-ENT stand out as measuring
things differently than Euclidean distance.

% MEASURES IN ORDER: (bad matfile saving)
%array(['condent', 'binary', 'cos', 'eucl', 'lhalf', 'lhalf_delta', 'dent',
%       'thresh', 'thresh_delta', 'cos_delta', 'jdiff', 'kl', 'ddiff',
%       'eucl_delta', 'leven'], 
%      dtype='|S12')
\begin{table}[t]
\begin{small}
\begin{center}
\begin{tabular}{|l|c|c|c|c|c|} \hline
 & Eucl. & $d_{1/2}$ & Jensen & Delta diff. & D-ENT \\ \hline
Eucl. & $1$ & $0.90$ & $0.84$ & $0.21$ & $0.12$ \\ 
$d_{1/2}$. & $0.90$ & $1$ & $0.71$ & $0.22$ & $0.27$ \\ 
Jensen & $0.84$ & $0.71$ & $1$ & $-0.04$ & $0.20$ \\ \hline 
Delta diff. & $\mathbf{0.21}$ & $0.22$ & $-0.04$ & $1$ & -$0.02$ \\ 
D-ENT & $\mathbf{0.12}$ & $0.27$ & $0.20$ & $-0.02$ & $1$ \\ \hline
cos & $0.88$ & $0.76$ & $0.97$ & $-0.03$ & $0.20$ \\
KL & $0.77$ & $0.64$ & $0.95$ & $-0.04$ & $0.17$ \\ 
cond. ent. & $0.58$ & $0.68$ & $0.37$ & $0.32$ & $0.18$ \\
thresh. & $0.68$ & $0.91$ & $0.51$ & $0.20$ & $0.35$ \\ \hline
% & eucl. & KL & D-ENT & $d_{1/2}$ & Jensen \\ \hline
%eucl. & $1$ & $0.88$ & $0.77$ & $0.12$ & $0.90$ & $0.84$\\
%cos &  & $1$ & $0.91$ & $0.20$ & $0.76$ & $0.97$ \\
%KL &  &  & $1$ & $0.17$ & $0.64$ & $0.95$ \\
%D-ENT &  & $$ & $$ & $1$ & $0.27$ & $0.20$ \\
%$d_{1/2}$ & & & & & $1$ & $0.71$ \\ 
%Jensen & & & & & & $1$ \\ 
%Leven & $0.59$ & $0.46$ & $0.35$ &$0.41$ & $0.81$ & $0.40$ \\
%Thresh. & $0.68$& $0.56$ & $0.46$ & $0.35$ & $0.90$ & $0.51$ \\ \hline
\end{tabular}
\caption{Pearson correlation between measures. Based on all results
from random, average, $1$NN and linear transform on $5000$K songs
with $1$ and $10$ missing beats. $1$ or $-1$ means high
correlation, $0$ means none.
Results form a symmetric matrix, we only show the upper triangle.
\label{tab:corrs}}
\end{center}
\end{small}
\end{table}

We take a closer look at this divergence between measures in Figure
\ref{fig:2dscore}.  It shows the performance of $3$ methods for
different numbers of missing beats. We report Euclidean distance and
D-ENT.  Nearest neighbor method ($k=1$) creates a reconstruction with
a granularity similar to the original for all mask sizes. It is a
direct consequence of the fact that D-ENT approximately constant
throughout a song. The linear transform learns to minimize the
Euclidean distance and does it successfully. But as it can be seen
from the other measure and Figure \ref{fig:basic} $3$rd row, it is
done by a large amount of smoothing. The average reconstruction has
the same granularity than the linear transform, but due to less
smoothing (or a less intelligent one), it does not do as well with the
Euclidean distance.

\begin{figure}[t]
\begin{center}
\includegraphics[width=.9\columnwidth]{recon_score_in_2d_5k}
\end{center}
\caption{Reconstruction error for $3$ methods and different
number of masked beats. Errors are D-ENT and Euclidean
distance. In all cases, the larger the number of masked beats,
the higher the Euclidean distance. Lower left is better.
\label{fig:2dscore}}
\end{figure}

We know report results of a $15$ beat imputation on $5000$ songs in
Table \ref{tab:res}. The linear transform is a clear winner based
on Euclidean distance. As before, nearest neighbor's strength
is to preserve the texture of the original patch as can be seen
from his D-ENT score. We can not report all results, but they
are no serious surprises with other measures 
as can be expected from Table \ref{tab:corrs}.

It is a disappointment that powerful methods such as SIPLCA
and HMM did not perform better in our experiments. It
probably only means that more research is needed. HMM should
be extended to model more than one beat at a time. For SIPLCA,
the activation matrix (that tells when each patch is used in time)
should be better constrained via priors for instance.

It is impossible results for every algorithm, measure and number of masked columns.
We selected the most meaningful ones; the reader is referred to our 
website\footnote{Code available at: \url{http://www.columbia.edu/~tb2332/something}}
for data and code to reproduce the results.

\begin{table}[t]
\begin{small}
\begin{center}
\begin{tabular}{|l||c|c|c|} \hline
method & Euclidean & delta diff. & D-ENT \\ \hline
random & $0.168$ & $$ & $0.252$ \\
average & $0.079$ & $$ & $0.430$ \\ \hline
1NN & $0.072$ & $$ & $\mathbf{0.123}$ \\
codebook & & & \\ \hline
lin. trans. & $\mathbf{0.056}$ & $$ & $0.479$ \\
SIPLCA & & & \\ \hline
\end{tabular}
\caption{Results on $15$ missing beats by different methods
on $5000$ songs and measured using Euclidean distance and
D-ENT.
\label{tab:res}}
\end{center}
\end{small}
\end{table}

Below are specific examples of original / reconstruction pairs.

\begin{figure}[t]
\begin{center}
\includegraphics[width=.9\columnwidth]{original_recons}
\end{center}
\caption{Selected pairs of original patches ($1$st column)
and their reconstructions ($2$nd column). 
Method used and some error measures
reported on the right.
\label{fig:origrecon}}
\end{figure}



\iffalse
\begin{table}[t]
\begin{small}
\begin{center}
\begin{tabular}{l|c|c|c|c|c|}
\# beats  & 1 & 2 & 5 & 10 & 15 \\ \hline \hline
random & $0.166$ & $0.166$ & $0.167$ & $0.167$ & $0.168$  \\
rand. song & $0.115$ & $0.114$ & $0.114$ & $0.115$ & $0.115$  \\
average all & $0.057$ & $0.057$ & $0.057$ & $0.057$ & $0.058$ \\
average & $0.047$ & $0.053$ & $0.062$ & $0.065$ & $0.069$ \\ \hline
knn eucl & $0.048$ & $0.049$ & $0.055$ & $0.064$ &  $0.070$ \\
knn kl & $0.049$ & $0.050$ & $0.056$ & $0.066$ &  $0.071$ \\
lin. trans. & $\mathbf{0.044}$ & $\mathbf{0.047}$ & $\mathbf{0.051}$ & $\mathbf{0.053}$ & $\mathbf{0.055}$ \\
codebook & & & & &  \\
SIPLCA & & & & &  \\ \hline
\end{tabular}
\caption{Results based on euclidean distance on $43K$ songs.
Song has to be at least $70$ beats long. 
For ``average'', window is $2$ beats each side of the masked patch.
For ``knn eucl'' and ``knn kl'', window is $10$ beats each side of the masked patch.
For ``lin. trans.'', window is the $2$ previous beats.}
\label{tab:reseucl}
\end{center}
\end{small}
\end{table}

\begin{table}[t]
\begin{small}
\begin{center}
\begin{tabular}{l|c|c|c|c|c|}
\# beats & 1 & 2 & 5 & 10 & 15 \\ \hline \hline
random & $0.428$ & $0.450$ & $0.461$ & $0.461$ & $0.462$  \\
rand. song & $0.334$ & $0.351$ & $0.371$ & $0.377$ & $0.380$  \\
average all & $0.164$ & $0.175$ & $0.183$ & $0.187$ & $0.189$ \\ 
average & $0.121$ & $0.154$ & $0.194$ & $0.212$ &  $0.223$ \\ \hline
knn eucl & $\mathbf{0.116}$ & $0.136$ & $0.169$ & $0.212$ & $0.233$ \\
knn kl & $\mathbf{0.116}$ & $\mathbf{0.135}$ & $\mathbf{0.167}$ & $0.209$ & $0.229$ \\
lin. trans. & $0.141$ & $0.157$ & $0.170$ & $\mathbf{0.180}$ & $\mathbf{0.184}$ \\
codebook & & & & &  \\
SIPLCA & & & & &  \\ \hline
\end{tabular}
\caption{Results based on symmetric KL divergence on $43K$ songs.
See Table \ref{tab:reseucl} for the exact parameters used.}
\label{tab:reskl}
\end{center}
\end{small}
\end{table}
\fi

\section{CONCLUSION AND FUTURE WORK}
\label{sec:conclusion}
As mentioned, modifications of HMM and SIPLCA are ongoing
research. There is great hope that algorithms pretrained
on large additional data (another set of songs) will break
the curse of boring smoothed patterns.
A unified set of measures should also be selected
by the community. Our code and test data is available to 
reproduce and improve these results.




%\section{ACKNOWLEDGEMENTS}
%NSERC PG grant for Thierry, something for Ron, NSF from Dan.


% References should be produced using the bibtex program from suitable
% BiBTeX files (here: strings, refs, manuals). The IEEEbib.bst bibliography
% style file from IEEE produces unsorted bibliography list.
% -------------------------------------------------------------------------
\bibliographystyle{IEEEbib}
\bibliography{tbm_bib}

\end{document}
