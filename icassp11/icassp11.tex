% Template for ICASSP-2010 paper; to be used with:
%          spconf.sty  - ICASSP/ICIP LaTeX style file, and
%          IEEEbib.bst - IEEE bibliography style file.
% --------------------------------------------------------------------------
\documentclass{article}
\usepackage{spconf,amsmath,graphicx}
\usepackage{url}

% Example definitions.
% --------------------
\def\x{{\mathbf x}}
\def\L{{\cal L}}

% Title.
% ------
\title{IMPUTATION OF BEAT-ALIGNED FEATURES AND MUSIC PATTERN LEARNING}
%
% Single address.
% ---------------
%\name{Thierry Bertin-Mahieux\thanks{Thanks to NSERC and some other stuff.}}
%\address{EE dept., Columbia University}
%
% For example:
% ------------
%\address{School\\
%	Department\\
%	Address}
%
% Two addresses (uncomment and modify for two-address case).
% ----------------------------------------------------------
\twoauthors
  {Thierry Bertin-Mahieux\sthanks{TBM is supported in part by a NSERC PG scholarship.}, Graham Grindlay,}
	{Columbia University\\
          LabROSA\\
	New York, USA}
  {Ron J. Weiss\sthanks{Thanks something} and Daniel P.W. Ellis}
	{New York University / Columbia University\\
          MARL / LabROSA\\
	New York, USA}

\begin{document}
%\ninept
%
\maketitle
%
\begin{abstract}
Imputation is cool. It is a well-defined problem, as opposed to segmentation.
Gives a reasonable benchmark to compare algorithms that claim learning meaningful
patterns on music. Hard to beat benchmarks comparison, for instance linear
prediction. We compare some methods and discuss contradicting error measures. 
We'll track you down if you don't accept this paper.
\end{abstract}
%
\begin{keywords}
Missing data, chroma features, beat imputation, codebook learning
\end{keywords}
%
\section{Introduction}
\label{sec:intro}
Many signals show similarities and repetitions over time. Learning these repetitions
in an unsupervised way remains a challenge. Finding robust patterns should prove useful in
many tasks like song similarity (songs with similar patterns), song segmentation
(song segments with different patterns) or cover song recognition (songs with exactly
the same patterns).
Therefore, we are looking for patterns that not only explain the signal (in an encoding
framework), but also contain musical characteristics. 

We devise a task that let us measure
the quality of such patterns in an attempt at unifying this research field.
Imputation is a well-studied problem for audio to show that some model can generalize.
However, it has not been pushed as far as it could, and fully masking consecutive beats
of music have not been considered. A real-life analogous problem would be a music stream
signal that is lost for a few seconds. The goal here would be to infer a reasonable
stream to replace the original one. Thus we do not only want a signal close to the original
in a reconstruction sense, but also a signal that has similar ``musical properties'' to
the user ear. These proporties are difficult to define, but we refer to higher order
dependancies present in the signal. For instance, the rate of note onsets, the number
of notes activated at the same time, and other texture-like elements. We will argue
that entropy is a good approximation of these characteristics.

Figure \ref{fig:types} summarizes the difference between different sorts of imputation.
Usually \cite{Smaragdis2009}, a few frequency bins are hidden and can be recovered from
their neighborhood, both in time and frequency. Masking one full beat is a natural extension,
and the solution is easy as music notes are often sustained over many beats. Two or three
missing beats remain solvable as music bare many repetitions. One can usually find a similar
section from an unmasked part of the song (think about a repeated chorus for instance).
The real challenge arises from masking many consecutive beats.

Figure \ref{fig:basic} shows an example of the imputation of $15$ beats on chroma features
(c.f. Subsection \ref{ssec:feats}). The solution in that figure, although intuitively wrong, is very
accurate according to euclidean distance. As we mentioned above, reconstruction error and
music properties do not agree. Understanding this conflict is essential to the task
definition.


\subsection{RELATED WORK} 
\label{ssec:relwork}

Prev work by me: \cite{Bertin-Mahieux2010a}. Imputation \cite{Smaragdis2009,Hoffman2010}.


\section{TASK DEFINITION}
\label{sec:task}

\subsection{DATA AND FEATURES}
\label{ssec:feats}
We use beat-aligned chroma features \cite{Ellis2007a}. Chroma features
record the intensity associated with each of the 12 semitones
(e.g. piano keys) within one octave, but all octaves
are folded together. One can see them as a very coarse and
noisy music transcription. They have been used for segmentation, 
chord recognition and cover recognition
among others. The specific implementation we use is the one
returned by the Echo Nest API \cite{EchoNest} which uses a constant-
Q spectrogram and where the chroma for each time stamp is
normalized so the max is 1. We use a random subset of $5K$ songs of
the cowbell dataset \cite{Bertin-Mahieux2010a} in our experiments.

\begin{figure}[t]
\begin{center}
\includegraphics[width=.7\columnwidth]{type_imputation}
\end{center}
\caption{Imputation types.
\label{fig:types}}
\end{figure}

\begin{figure}[t]
\begin{center}
\includegraphics[width=.95\columnwidth]{basic}
\end{center}
\caption{$15$ beats imputation example, rows are 1) original 2) original masked
3) reconstruction using a linear transform of one previous beat.
\label{fig:basic}}
\end{figure}

% IPYTHON COMMAND TO RECREATE ABOVE FIGURE
%btchroma2 = sio.loadmat('/home/thierry/Columbia/covers80/coversongs/covers32kENmats/john_lennon+Double_Fantasy+05-I_m_Losing_You.mp3.mat')['btchroma']
%p1=185;p2=p1+15;mask=np.ones(btchroma2.shape);mask[:,p1:p2]=0.

\subsection{MEASURES}
\label{ssec:measures}
The choice of a measure function is closely related to the properties
we expect to preserve in learned patterns. As we saw in Figure
\ref{fig:basic}, reconstruction error does not tell the whole story.
We present different error measures and look at what imputation they
favor.

We consider Mahalanobis distances of the form $d_p = |x_1-x_2|^p$, especially
with $p \leq 2$. Figure \ref{fig:measures} offers a reminder of the effect
of these measures on the absolute difference between a pixel and
its reconstruction. The middle square represents the case where a
reconstruction is considered valid if it is between some $\delta$ of
the original and wrong otherwise. Mahalanobis distance approximate
this case with an increasingly smaller $\delta$ as $p \rightarrow 0$.
The consequence can be seen on the toy example of 
Figure \ref{fig:square}. We approximate a square wave of period $1$
by two signals: the same square wave translated by $0.25$ and the
average function (constant at $0.5$). The average reconstruction 
errors on $[0,1]$ for the translated wave, with any $d_p$, is $0.5$.
For the average and $d_2$, this error is $0.25$.
Hence, the average function seems the most accurate. However,
with $d_1$, the errors are both equal to $0.5$. With $d_{1/2}$,
errors are now $0.5$ and $0.71$. Hence, if the translated square wave
should be a better approximation, euclidean distance is misleading.

Vasicek estimator \cite{Learned-Miller2003}.
Jensen difference \cite{Michel1994}.

The phenomena above explain why reconstruction such as
in Figure \ref{fig:basic} ($3$rd row) or in Figure \ref{fig:avgnnrand}
($3$rd row) often result from powerful algorithms. An intuitive
conclusion is that we must penalize sustained beat solutions.
An idea is to measure the difference in the delta-features, i.e.
the discrete difference in time of the original features.
Another possibility is to use entropy-based measures as introduced
in \cite{Mentzelopoulos2004}.
Absolute normalize difference entropy (D-ENT)
between two vectors is computed as follow. First, compute the normalized histogram 
of values, in our case we discretize $(0,1)$ in $\mbox{\# bins} = 10$. 
Then, we compute the entropy
$e_b = - log_2 b_b$ of each of the bins $b_{bi}$ for each of the two vectors $i$. Finally:
\[
\mbox{D-ENT} = \left( \Sigma_b \frac{e_{b1} - e_{b2}}{e_{b1}} \right) / (\mbox{\# bins})
\]
Note that D-ENT is not symmetric. In our experiment, the first vector is the original one.
If we look at Figure \ref{fig:avgnnrand}, the reconstruction with the lowest D-ENT is using
nearest neighbor and not averaging as with euclidean distance.

Surprisingly, Kullback-Leibler divergence (KL) that is entropy related does not seem
to measures things that differently from euclidean distance in our case. See Table
\ref{tab:corrs}. Similar disappointment for $d_{1/2}$.

We have explained why euclidean distance justifies disappointing reconstructions. At the same
time, we do not argue that we should ignore or replace it. Euclidean distance measures
reconstruction in a fundamental way. We believe we need a set of measures to quantify the
quality of music patterns, euclidean distance being one of them.

\begin{figure}[t]
\begin{center}
\includegraphics[width=.95\columnwidth]{avg_nn_rand}
\end{center}
\caption{Same beat imputation example as Figure \ref{fig:basic}, 
rows are 1) original 2) original masked
3) reconstruction using the average of nearby beats 4) using
nearest neighbor 5) using random.
\label{fig:avgnnrand}}
\end{figure}

\begin{figure}[t]
\begin{center}
\includegraphics[width=.85\columnwidth]{measures}
\end{center}
\caption{Effect of different measures on the difference between a pixel
and its reconstruction.
\label{fig:measures}}
\end{figure}

\begin{figure}[t]
\begin{center}
\includegraphics[width=.8\columnwidth]{square}
\end{center}
\caption{Reconstruction error between a square wave and two approximation,
a square wave translated by a quarter of the period, and the average
function. Average error between original and translated wave is always $0.5$
for any Mahalanobis measure $d_p$ on $[0,1]$. 
For the average function, the errors are
$0.25$, $0.5$ and $0.71$ for $d_2$, $d_1$ and $d_{1/2}$ respectively.
\label{fig:square}}
\end{figure}

\section{EXPERIMENTS}
\label{sec:exp}
We present results from different algorithms for imputing masked beats.
Features are beat-aligned chromas and the dataset is made of $5000$ songs
from the Cowbell dataset. Songs has to be of length at least $100$ beats.
The masked part is chosen at random, excluding the first and last $30$ beats.

\subsection{ALGORITHMS}
\label{ssec:algo}
Algorithms named in bold are the ones used in the experiments.

Simple methods include \textbf{random} where each pixel is drawn from a uniform $[0,1)$
distribution (Figure \ref{fig:avgnnrand}, $5$th row). 
We can also pick a beat at random from the song, or take the average
of all visible beats. A more efficient way is to take the \textbf{average} of the beats
within a certain range of the missing patch, for example within $3$ beats on each
side (Figure \ref{fig:avgnnrand}, $3$rd row).
A significant improvement is to predict the next missing beat based on the
previous $N$ ones. We train a \textbf{linear transform} by minimizing the euclidean distance
on the visible parts of the song. This methods is shown in Figure \ref{fig:basic}.

Nearest neighbor ($\mathbf{k}$\textbf{NN}) is a powerful technique. 
As we mentioned in the introduction, many segments
of a song are repeated at least once. Thus, we can infer the missing beats by looking at
the neighbor ones and finding another section that has similar adjacent beats. Note
that we can use a \textbf{codebook} made of other songs instead of the current one. Also, future
work include working with $k > 1$.

We experimented with algorithms with greater representational power, e.g. \textbf{HMM} and
shift invariant probabilistic latent component analysis (\textbf{SIPLCA}) \cite{Smaragdis2009,Weiss2010}.
ADD SOME DETAILS, DIAGONAL/SEQUENCE STATE TRANSITION MATRIX FOR THE HMM, 
RANK AND WIN SIZE FOR SIPLCA.
Due to lack of space and the numerous parameters to tune, we present results without
much details. We refer the reader to our code and mention that this is part of our
ongoing research.


\subsection{RESULTS}
\label{ssec:results}
It is impossible results for every algorithm, measure and number of masked columns.
We selected the most meaningful ones; the reader is referred to our 
website\footnote{Code available at: \url{http://www.columbia.edu/~tb2332/something}}
for data and code to reproduce the results.

Figure \ref{fig:2dscore} shows the performance of $3$ methods for different numbers
of missing beats. We report euclidean distance and D-ENT. One can see the result
of contradicting measures. Nearest neighbor method ($k=1$) creates a reconstruction
with an entropy similar to the original for all mask sizes. It is a direct consequence
of the fact that entropy is approximately constant throughout a song. The linear
transform learns to minimize the euclidean distance and does it successfully. But as it
can be seen from its entropy measure and Figure \ref{fig:basic} $3$rd row, it is
done by a large amount of smoothing. The average reconstruction has the same entropy
results than the linear transform, but due to less smoothing (or a less intelligent
one), it does not do as well with the euclidean distance.

\begin{figure}[t]
\begin{center}
\includegraphics[width=.9\columnwidth]{recon_score_in_2d_5k}
\end{center}
\caption{Reconstruction error for $3$ methods and different
number of masked beats. Errors are D-ENT and euclidean
distance. In all cases, the larger the number of masked beats,
the higher the euclidean distance. Lower left is better.
\label{fig:2dscore}}
\end{figure}

In Subsection \ref{ssec:measures} we mentioned many different measures.
We also showed how they could favor different kinds of reconstruction
(with heavy smoothing or not, for instance). To make some sense of
these, we investigate over large set of experiments which measures tend
to agree with each other. We use Pearson's 
correlation\footnote{
$\rho_{X,Y}=\mathrm{corr}(X,Y)={\mathrm{cov}(X,Y) \over \sigma_X \sigma_Y} ={E[(X-\mu_X)(Y-\mu_Y)] \over \sigma_X\sigma_Y}$
} $\rho_{X,Y}$. We have $-1 \leq \rho_{X,Y} \leq 1$, an absolute
value close to $1$ meaning high correlation.
$\rho_{X,Y}$ is computed for many error measures and results are presented in
Table \ref{tab:corrs}. It appears that measures tend to form two groups.
The ``euclidean-like'' measures favor smoothing to minimize divergence on
average. The ``entropy-like'' measures look at the texture of the reconstruction
compared to the original (ONLY D-ENT IN THAT CATEGORY?????).

\begin{table}[t]
\begin{small}
\begin{center}
\begin{tabular}{|l|c|c|c|c|c|c|} \hline
 & eucl. & cos & KL & D-ENT & $d_{1/2}$ & Jensen \\ \hline
eucl. & $1$ & $0.88$ & $0.77$ & $0.12$ & $0.90$ & $0.84$\\
cos &  & $1$ & $0.91$ & $0.20$ & $0.76$ & $0.97$ \\
KL &  &  & $1$ & $0.17$ & $0.64$ & $0.95$ \\
D-ENT &  & $$ & $$ & $1$ & $0.27$ & $0.20$ \\
$d_{1/2}$ & & & & & $1$ & $0.71$ \\ 
Jensen & & & & & & $1$ \\ \hline
\end{tabular}
\caption{Pearson correlation between measures. Based on all results
from random, average, $1$NN and linear transform on $5000$K songs
with $1$ and $10$ missing beats. $1$ or $-1$ means high
correlation, $0$ means none.
Results form a symmetric matrix, we only show the upper triangle.
\label{tab:corrs}}
\end{center}
\end{small}
\end{table}

We know report results of a $15$ beat imputation on $5000$ songs in
Table \ref{tab:res}. The linear transform is a clear winner based
on euclidean distance. As before, nearest neighbor's strength
is to preserve the texture of the original patch as can be seen
from his D-ENT score. We can not report all results, but they
are no serious surprises with other measures 
as can be expected from Table \ref{tab:corrs}.

It is a disappointment that powerful methods such as SIPLCA
and HMM did not perform better in our experiments. It
probably only means that more research is needed. HMM should
be extended to model more than one beat at a time. For SIPLCA,
the activation matrix (that tells when each patch is used in time)
should be better constrained via priors for instance.

\begin{table}[t]
\begin{small}
\begin{center}
\begin{tabular}{|l||c|c|c|} \hline
method & euclidean & KL & D-ENT \\ \hline
random & $0.168$ & $0.461$ & $0.252$ \\
average & $0.079$ & $0.249$ & $0.430$ \\ \hline
1NN & $0.072$ & $0.237$ & $\mathbf{0.123}$ \\
codebook & & & \\ \hline
lin. trans. & $\mathbf{0.056}$ & $0.183$ & $0.479$ \\
SIPLCA & & & \\
HMM & & & \\ \hline
\end{tabular}
\caption{Results on $15$ missing beats by different methods
on $5000$ songs and measured using euclidean distance and
D-ENT.
\label{tab:res}}
\end{center}
\end{small}
\end{table}

\iffalse
\begin{table}[t]
\begin{small}
\begin{center}
\begin{tabular}{l|c|c|c|c|c|}
\# beats  & 1 & 2 & 5 & 10 & 15 \\ \hline \hline
random & $0.166$ & $0.166$ & $0.167$ & $0.167$ & $0.168$  \\
rand. song & $0.115$ & $0.114$ & $0.114$ & $0.115$ & $0.115$  \\
average all & $0.057$ & $0.057$ & $0.057$ & $0.057$ & $0.058$ \\
average & $0.047$ & $0.053$ & $0.062$ & $0.065$ & $0.069$ \\ \hline
knn eucl & $0.048$ & $0.049$ & $0.055$ & $0.064$ &  $0.070$ \\
knn kl & $0.049$ & $0.050$ & $0.056$ & $0.066$ &  $0.071$ \\
lin. trans. & $\mathbf{0.044}$ & $\mathbf{0.047}$ & $\mathbf{0.051}$ & $\mathbf{0.053}$ & $\mathbf{0.055}$ \\
codebook & & & & &  \\
SIPLCA & & & & &  \\ \hline
\end{tabular}
\caption{Results based on euclidean distance on $43K$ songs.
Song has to be at least $70$ beats long. 
For ``average'', window is $2$ beats each side of the masked patch.
For ``knn eucl'' and ``knn kl'', window is $10$ beats each side of the masked patch.
For ``lin. trans.'', window is the $2$ previous beats.}
\label{tab:reseucl}
\end{center}
\end{small}
\end{table}

\begin{table}[t]
\begin{small}
\begin{center}
\begin{tabular}{l|c|c|c|c|c|}
\# beats & 1 & 2 & 5 & 10 & 15 \\ \hline \hline
random & $0.428$ & $0.450$ & $0.461$ & $0.461$ & $0.462$  \\
rand. song & $0.334$ & $0.351$ & $0.371$ & $0.377$ & $0.380$  \\
average all & $0.164$ & $0.175$ & $0.183$ & $0.187$ & $0.189$ \\ 
average & $0.121$ & $0.154$ & $0.194$ & $0.212$ &  $0.223$ \\ \hline
knn eucl & $\mathbf{0.116}$ & $0.136$ & $0.169$ & $0.212$ & $0.233$ \\
knn kl & $\mathbf{0.116}$ & $\mathbf{0.135}$ & $\mathbf{0.167}$ & $0.209$ & $0.229$ \\
lin. trans. & $0.141$ & $0.157$ & $0.170$ & $\mathbf{0.180}$ & $\mathbf{0.184}$ \\
codebook & & & & &  \\
SIPLCA & & & & &  \\ \hline
\end{tabular}
\caption{Results based on symmetric KL divergence on $43K$ songs.
See Table \ref{tab:reseucl} for the exact parameters used.}
\label{tab:reskl}
\end{center}
\end{small}
\end{table}
\fi

\section{CONCLUSION AND FUTURE WORK}
\label{sec:conclusion}
As mentioned, modifications of HMM and SIPLCA are ongoing
research. There is great hope that algorithms pretrained
on large additional data (another set of songs) will break
the curse of boring smoothed patterns.
A unified set of measures should also be selected
by the community. Our code and test data is available to 
reproduce and improve these results.


\section{ACKNOWLEDGEMENTS}
Your mama.
NSERC PG grant for Thierry, something for Ron, NSF from Dan.


% References should be produced using the bibtex program from suitable
% BiBTeX files (here: strings, refs, manuals). The IEEEbib.bst bibliography
% style file from IEEE produces unsorted bibliography list.
% -------------------------------------------------------------------------
\bibliographystyle{IEEEbib}
\bibliography{tbm_bib}

\end{document}
